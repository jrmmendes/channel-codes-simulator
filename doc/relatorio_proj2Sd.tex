\documentclass[]{article}

%Informações do documento
\title{ An{\'a}lise de desempenho para c{\'o}digos de canal}
\author{Jos{\'e} Romildo, Thales Henrique, Railton Rocha}


\usepackage[brazil]{babel}
\usepackage[utf8]{inputenc}
\usepackage{geometry}
\usepackage{indentfirst}
\usepackage{setspace}
\usepackage{makeidx}
\usepackage{hyperref}

%formatação da página
\geometry{
	a4paper,
	total={170mm,257mm},
	left=20mm,
	top=20mm,
}

\onehalfspacing

\makeindex
\begin{document}
\maketitle

\index{1}
\section{Introdução}

\par
Os códigos de tratamento de erros são de grande importância nos sistemas de comunicação modernos. Com efeito, a utilização dos mesmos pode ser a diferença entre aqueles que são ou não funcionais, uma vez que é possível detectar e, possivelmente, corrigir erros em mensagens sem a necessidade de retransmissão dos dados. 
...

Este relatório está organizado da maneira que se segue. Na Seção 2 é apresentada toda a base teórica referente a álgebra abstrata e teoria de códigos utilizada no projeto, bem como uma pequena revisão acerca dos canais BSC. A Seção 3 apresenta a metodologia utilizada nas simulações, sendo estas mostradas na Seção 4. Na Seção 5 os resultados são analizados, e o relatório é concluido na Seção 6.



\section{Base Teórica}
\subsection{Grupos}
\subsection{Subgrupos e Teorema de Lagrange}
\subsection{Espaços Vetoriais}
\subsection{Códigos de Tratamento de Erros}
\subsection{Binary Simetric Channel (BSC)}

\section{Metologia}
\section{Simulações}

\section{Análise dos Resultados}

\section{Conclusão}

\appendix
\section{Apêndice}
\subsection{Código-fonte do simulador}

\end{document}
