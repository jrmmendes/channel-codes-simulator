\documentclass[]{article}

%Informações do documento
\title{ An{\'a}lise de desempenho para c{\'o}digos de canal}
\author{Jos{\'e} Romildo, Thales Henrique, Railton Rocha}


\usepackage[brazil]{babel}
\usepackage[utf8]{inputenc}
\usepackage{geometry}
\usepackage{indentfirst}
\usepackage{setspace}
\usepackage{makeidx}
\usepackage{hyperref}
\usepackage{enumerate}
\usepackage{amsfonts}

%formatação da página
\geometry{
	a4paper,
	total={170mm,257mm},
	left=20mm,
	top=20mm,
}

\onehalfspacing

\makeindex
\begin{document}
\maketitle

\index{1}
\section{Introdução}

\par
Os códigos de tratamento de erros são de grande importância nos sistemas de comunicação modernos. Com efeito, a utilização dos mesmos pode ser a diferença entre aqueles que são ou não funcionais, uma vez que é possível detectar e, possivelmente, corrigir erros em mensagens sem a necessidade de retransmissão dos dados. 
... %necessário complementar

\par
Este relatório está organizado da maneira que se segue. Na Seção 2 é apresentada toda a base teórica referente a álgebra abstrata e teoria de códigos utilizada no projeto, bem como uma pequena revisão acerca dos canais BSC. A Seção 3 apresenta a metodologia utilizada nas simulações, sendo estas mostradas na Seção 4. Na Seção 5 os resultados são analizados, e o relatório é concluido na Seção 6.

\section{Base Teórica}
\subsection{Grupos}

\par
Seja $G$ um conjunto. Uma operação binária em $G$ é uma função que atribui, a cada par de elementos em $G$, um outro elemento em $G$ . A estrutura algébrica  $\langle G , * \rangle$ , em que $*$ é uma operação binária defina em G , é um grupo se, $\forall g , h , k \in G,$

\begin{enumerate}[G1.]
	\item $g * h \in G$ (fechamento)
	\item $ g * (h * k) = (g*h)*k$ (associatividade)
	\item $ \exists e \in G ; e * g = g * e = g $ (identidade)
	\item $ \exists g^{-1} \in G;  g*g^{-1} = g^{-1} *g=e $ (inverso)
\end{enumerate}

\par
Se, além das propriedades acima, for válida a comutatividade ( $ g*h = h*g , \forall g , h \in G$ ), então o grupo é dito abeliano ou comutativo. Também é possível classificar essa estrutura pela sua cardinalidade, podendo existir grupos finitos ou infinitos.
Se somente o fechamento vale, têm-se um grupoide; se, além deste, a associatividade também é verificada , um semigrupo. Por fim, no caso em que apenas a existência do inverso não é satisfeita, têm-se um monoide.

\par
No que diz respeito a grupos, ainda é possível destacar algumas propriedades. Se $h$ e $g$ são elementos de um grupo $G$, 

\begin{enumerate}[PG1.]
	\item O elemento identidade é único;
	\item O elemento inverso de um elemento $g$ é único;
	\item  O inverso do elemento $k = g*h$ é $k^{-1}=g^{-1}*h-1$ ;
	\item Todo elemento é cancelável, i.e., se $h _ 1 * g = h _ 2 * g$ e $g*h _ 1 = g*h _ 2$ , então $h _ 1 = h _ 2$ ;
	\item Para quaisquer elementos $a$ e $b$ do grupo, a equação $a*x =b$ tem uma única
	solução.
\end{enumerate}

\par
O número de elementos de um grupo, seja ele finito ou infinito, é chamado de ordem do grupo. A ordem do grupo $G$ é denotada usando o símbolo de valor absoluto, i.e., por $\left| G \right|$.Por exemplo, o grupo dos inteiros $\mathbb{Z}$ sobre a adição tem uma ordem infinita (número infinito de elementos), enquanto o grupo $U(10) = \lbrace 1,3,7,9 \rbrace$ sobre a multiplicação módulo 10 tem  ordem 4.

Já no que se refere aos elementos, a ordem de $g$, se $e$ é o elemento identidade, é o menor inteiro positivo $n$ para o qual $g^n = e$, ou, em notação aditiva, $ng = e$. Se tal número não existe, e dito que a órdem $\left|g\right|$ é infinita. 

\subsection{Subgrupos}

\par
Se um subconjunto $H$ de um grupo $G$ é ele próprio um grupo sobre a operação definida em $G$, então é dito ser subgrupo deste. Essa relação é denotada por $H \leq G $ (lê-se “$H$ é subgrupo de $G$). Todo grupo é subgrupo dele mesmo. Se $H$ é subgrupo de $G$, mas não é igual a $G$, então ele é chamado de \textbf{subgrupo próprio} (\textit{proper subgroup}) e pode ser usada a notação $H < G$. Além disso, o subgrupo trivial $\lbrace e \rbrace$ é subgrupo de qualquer grupo. Os demais que eventualmente existam são ditos subgrupos não-triviais de $G$. No Apêndice A são mostrados alguns métodos de testes para subgrupos.

\subsection{Classes Laterais e Teorema de Lagrange}

\par
O Teorema de Lagrange é sem dúvidas um dos mais importantes resultados dentro da teoria dos grupos finitos, considerado, sobretudo, o “ABC da Álgebra Abstrata”. Para provar sua validade, é necessário estudar antes uma ferramenta matemática, as chamadas \textbf{Classes Laterais} (ou \textit{Coset’s}).



\subsection{Espaços Vetoriais}
\subsection{Códigos de Tratamento de Erros}
\subsection{Binary Simetric Channel (BSC)}

\section{Metologia}
\section{Simulações}

\section{Análise dos Resultados}

\section{Conclusão}

\appendix
\section{Testes de Subgrupo}
\section{Código-fonte do simulador}

\end{document}
